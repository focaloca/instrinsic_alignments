% intrinsic size and convergence correlations in weak lensing
% Basundhara Ghosh, Ruth Durrer, Bjoern Malte Schaefer

\documentclass[a4paper,fleqn,usenatbib]{mnras}
\usepackage[T1]{fontenc}
\usepackage{ae,aecompl}
\usepackage{graphicx}
\usepackage{amsmath}
\usepackage{amssymb}
\usepackage{txfonts}

\usepackage{float}
\usepackage{adjustbox}

\graphicspath{ {images/} }

\def\spirou#1{{\bf #1}}
\def\foca#1{{\bf #1}}
\def\basundhara#1{{\bf #1}}
\def\ruth#1{{\bf #1}}


% --- macros --- %
\newcommand{\icm}{intra-cluster medium}
\newcommand{\cmb}{cosmic microwave background}


% --- spirou's commands --- %
\newcommand{\ltsima}{$\; \buildrel < \over \sim \;$}
\newcommand{\lsim}{\lower.5ex\hbox{\ltsima}}
\newcommand{\gtsima}{$\; \buildrel > \over \sim \;$}
\newcommand{\gsim}{\lower.5ex\hbox{\gtsima}}
\newcommand{\bra}{\langle}
\newcommand{\ket}{\rangle}
\newcommand{\dang}{d_\mathrm{A}}
\newcommand{\dd}{\mathrm{d}}
\newcommand{\e}{\mathrm{e}}
\newcommand{\p}{\mathrm{p}}

\newcommand{\ci}{\mathrm{i}}
\newcommand{\vecx}{\bmath{x}}
\newcommand{\geo}{\vecx(\btheta,\chip)}
\newcommand{\vecl}{\bmath{\ell}}
\newcommand{\veclp}{\bmath{\ell}^\prime}
\newcommand{\trace}{\mathrm{tr}}
\newcommand{\dlp}{(\ell-\lprime)}

\newcommand{\chip}{{\chi^\prime}}
\newcommand{\chipp}{{\chi^{\prime\prime}}}
\newcommand{\lprime}{\ell^\prime}
\newcommand{\dirac}{\delta_D}
\newcommand{\likeli}{\mathcal{L}}
\newcommand\BG[1]{\textcolor{red}{#1}}

\onecolumn



% --- title --- %
\title[Intrinsic sizes and shapes of galaxies]
{Intrinsic and extrinsic correlations of galaxy shapes and sizes in weak lensing data}
\author[B. Ghosh, R. Durrer, B.M. Sch{\"a}fer]
{Basundhara Ghosh$^1$, Ruth Durrer$^1$, Bj{\"o}rn Malte Sch{\"a}fer$^2$\thanks{e-mail: bjoern.malte.schaefer@uni-heidelberg.de}\\
$^1$D{\'e}partment de la Physique Th{\'e}orique, Universit{\'e} de Gen{\`e}ve, 24 quai Ernest Ansermet, 1211 Gen{\`e}ve, Switzerland\\
$^2$Zentrum f{\"u}r Astronomie der Universit{\"a}t Heidelberg, Astronomisches Rechen-Institut, Philosophenweg 12, 69120 Heidelberg, Germany
}


% --- document --- %
\begin{document}
\pagerange{\pageref{firstpage}--\pageref{lastpage}}
\pubyear{2020}
\maketitle
\label{firstpage}


% --- abstract --- %
\begin{abstract}
The subject of this paper are shape and size correlations of galaxies due to weak gravitational lensing and due to direct tidal interaction of elliptical galaxies with gravitational fields sourced by the cosmic large-scale structure. Setting up a linear intrinsic alignment model that is able to predict intrinsic shape and size correlations in a consistent way we compute the spectra of the intrinsic correlations as well as of the cross-correlations between intrinsic shapes and sizes with weak gravitational shear and convergence, juxtaposing both types of spectra with extrinsic shapes and sizes correlations caused by lensing. We quantify the observability of the intrinsic shape and size correlations and quantify with the Fisher-formalism how well the alignment parameter can be determined from the Euclid weak lensing survey. Specifically, we find a contamination of the weak lensing convergence spectra with an intrinsic size correlation amounting to \spirou{add number}\% over a wide multipole range $\ell=30\ldots300$, with a corresponding cross-correlation exhibiting a sign change, similar to the cross-correlation between weak lensing shear and intrinsic shapes. Quantifying the information content of size correlations in addition to shape correlations shows that bounds can be improved by \spirou{add number}\% for a $w$CDM-cosmology, and that intrinsic correlations decrease the sensitivity of weak lensing due to the negative sign of the cross-correlations by \spirou{add number}. A determination of the alignment parameter yields an precision of $\sigma_D=$\spirou{add number} forecasted for Euclid.
\end{abstract}


% --- keywords --- %
\begin{keywords}
gravitational lensing: weak -- dark energy -- large-scale structure of Universe.
\end{keywords}


% --- section:  --- %
\section{introduction}\label{sect_intro}
Weak lensing has emerged as a powerful probe for investigating the cosmic large-scale structure, for testing gravitational theories and for constraining cosmological parameters. As gravitational lensing probes fluctuations in the gravitational potential directly, it depends on minimal assumptions and is fixed for a given gravitational theory. Correlations in the shapes of galaxies induced by weak lensing have been detected almost two decades ago, and by now lensing is recognised as a tool for investigating cosmological theories alongside the cosmic microwave background and galaxy clustering. The last generation of surveys, most notably KiDS and DES \citep{Abbott:2017wau} have provided independent confirmation for the $\Lambda$CDM-model and support parameter determinations from the CMB, even though tensions between the two probes, most notably in the matter density $\Omega_m$ and $\sigma_8$ remain. The next generation of surveys, in particular Euclid \citep{Amendola:2016saw} and LSST will probe cosmological models to almost fundamental limits of cosmic variance, but with decreasing statistical errors the control of systematical errors will become one of the central questions for data analysis, along with higher-order effects in the lensing signal related to evaluating the tidal shear fields along a geodesic \citep{Ghosh:2018nsm}, as well as non-Gaussian statistics of the lensing signal due to nonlinear structure formation and non-Gaussian contributions to the covariance.

Among astrophysical contaminants of the weak lensing signal, intrinsic alignments are perhaps the most dramatic, leading to significant biases in the estimation of cosmological parameters. There are two primary models for the two dominant galaxy types for linking the apparent shapes to tidal gravitational fields in the large-scale structure, which acts, due to long-ranged correlations, as the medium to reduce randomness and to correlate the measured ellipticities. The shapes of spiral galaxies are thought to be determined by the orientation of the angular momentum of the stellar disc, and ultimately of the dark matter halo harbouring the stellar component. With this idea in mind, shape correlations are traced back to angular momentum correlations, which in turn would depend through tidal torquing as the angular momentum generated mechanism on the tidal shear fields. Tidal torquing models commonly predict ellipticity correlations on small scales at a level of at most 10\% of the weak lensing signal on multipoles above $\ell\simeq300$ for a survey like Euclid, many physical assumptions have been challenged, most notably the orientation of the disc relative to the host halo angular momentum, as well as an overprediction of the correlation inherent to the torquing mechanism.

Elliptical galaxies, on the other hand, are thought to acquire shape correlations through direct interaction with the tidal shear field: Second derivatives of the gravitational potential would give rise to an anisotropic deformation of the galaxy, in the principal directions of the tidal shear tensor. Interestingly, the reaction of a galaxy to the tidal shear field is determined by the inverse velocity dispersion $1/\sigma^2$ similar to lensing, where the relevant quantity is the gravitational potential in units of $c^2$. Tidal alignments of elliptical galaxies are thought to be present at intermediate angular scales of a few hundred in multipole $\ell$ for a survey like Euclid, with amplitudes being typically an order of magnitude smaller than that of the weak lensing effect. In parallel, alignment models using ideas from effective field theories provide parameterised relationships between tensors constructed from the cosmic density and velocity fields and can capture a wider range of alignment mechanisms and track them into the nonlinear regime, but perhaps with a less clear physical picture. 

While intrinsic alignments refer to a physical change of the appearance of the galaxies, there is an analogous deformation effect on the shape of the light bundle emanating from a galaxy by gravitational lensing. To lowest order, both effects depend on tidal gravitational field which suggests that the effects must be correlated. In fact, cross-correlations between the physical change in shape and the apparent change in shape are predicted to be nonzero for elliptical galaxies, and to be more exact, should in fact be negative as galaxies align themselves radially with a large structure while lensing generates a tangential alignment. As a result, ellipticity correlations of galaxies is a sum of the conventional weak lensing (referred to as GG), the intrinsic alignment (or II) and the cross-correlation between the two (called GI).

There should be analogous effects of the size of an elliptical galaxy due to tidal gravitational fields: In gravitational lensing the light bundle can be isotropically enlarged, i.e. changed in size while the shape is conserved: This nonzero convergence is caused by the trace of the tidal shear field. Similarly, the size of an elliptical galaxy would physically change for a fixed velocity dispersion if the trace of the tidal shear field is nonzero, or equivalently, if it resides in an overdense or underdense region. An underdense region with density contrast $\delta < 0$ would source a gravitational potential $\Phi$ through the Poisson-equation $\Delta\Phi/c^2 = 3\Omega_m/(2\chi_H^2)\delta$, with the Hubble-distance $\chi_H = c/H_0$, such that the eigenvalues of $\partial_i\partial_j$ would be negative, stretching the galaxy to a larger size.

The motivation of our paper are exactly these correlations between the sizes of elliptical galaxies as they would be predicted by a linear alignment model as a consequence of the trace $\Delta\Phi$ of the tidal shear tensor $\partial_i\partial_j\Phi$ being nonzero. These intrinsic size correlations would be generated in complete analogy to intrinsic shape correlations caused by the traceless part of the tidal shear, and would contaminate measurements of weak lensing convergence correlations in the same way as intrinsic shape correlations are a nuisance to the weak lensing shear.

After introducing tidal interactions of elliptical galaxies with their surrounding large-scale structure in Sect.~\ref{sect_tidal}, we compute shape correlations from direct tidal interaction and through gravitational lensing in Sect.~\ref{sect_shapes} as well as the corresponding size correlations in Sect.~\ref{sect_sizes}. We quantify the information content of each of the correlations and the amount of covariance in Sect.~\ref{sect_fisher}, before discussing our results in Sect.~\ref{sect_summary}. In general we work in the context of a $w$CDM-cosmology with a constant equation of state value of $w$ close to $-1$, and standard values for the cosmological parameters, i.e. $\Omega_m = 0.3$, $\sigma_8 =  0.8$, $h = 0.7$ and $n_s = 0.96$, and a parameterised spectrum for nonlinearly evolving scales.


% --- section:  --- %
\section{tidal interactions of galaxies and gravitational lensing}\label{sect_tidal}
In a simplified way one can imagine elliptical galaxies as a stellar component in virial equilibrium with a velocity dispersion $\sigma^2$, filling the gravitational potential. If the gravitational potential is distorted by external fields as the galaxy is not an isolated object, the equipotential contours get distorted and correspondingly, the stellar component reacts and galaxy assumes a different shape. To lowest order, the change in shape takes place along the principal axes of the tidal shear tensor $\partial_i\partial_j\Phi$, which is defined as the second derivatives of the gravitational potential $\Phi$. Consequently, one observes a complex ellipticity $\epsilon$,
\begin{equation}
\epsilon = \epsilon_+ + \ci\epsilon_\times \propto 
(\partial_x\partial_x-\partial_y\partial_y)\Phi +2\ci\partial_x\partial_y\Phi
\end{equation}
with a constant of proportionality $D$, which describes the elasticity of the stellar component: As shown by \spirou{add reference}, in is inversely proportional to the velocity dispersion $\sigma^2$. With many galaxies in a tomographic bin $A$ with a suitable, normalised redshift distribution $p_A(z)\dd z$ one can define the line of sight-averaged ellipticity from second angular derivatives of the weighted projection of the potential $\Phi$:
\begin{equation}
\varphi_{A,ab} = \partial_a\partial_b\varphi_A
\quad\mathrm{with}\quad
\varphi_A = \int\dd\chi\:p_A(z(\chi))\frac{c}{H(\chi)}\:\Phi = \int\dd\chi\:W_{\varphi,A}(\chi)\:\Phi,
\end{equation}
with the Hubble-function $H(\chi)$. The angular derivatives $\partial_a$ are related to the spatial derivatives $\partial_x$ through $\partial_a = \chi\partial_x$, with $x=\theta\chi$ in the small-angle approximation. From that, one can recover the ellipticity components $\epsilon_{+,A}$ and $\epsilon_{\times,A}$ as well as the size $s_A$ from a decomposition of the tensor $\varphi_{A,ab}$ with the Pauli-matrices $\sigma_{ab}^{(n)}$,
\begin{equation}
\varphi_{A,ab} = s_A\sigma^{0}_{ab} + \epsilon_{+,A}\sigma^{(1)}_{ab} + \epsilon_{\times,A}\sigma^{(3)}_{ab},
\mathrm{~with}\quad
\sigma^{(0)} = \left(
\begin{array}{cc}
1 & 0 \\ 0 & 1
\end{array}
\right),
\sigma^{(1)} = \left(
\begin{array}{cc}
1 & 0 \\ 0 & -1
\end{array}
\right),
\mathrm{~and}\quad
\sigma^{(3)} = \left(
\begin{array}{cc}
0 & 1 \\ 1 & 0
\end{array},
\right)
\end{equation}
where three components are sufficient because of the symmetry $\varphi_{A,ab} = \varphi_{A,ba}$. Using two properties of the Pauli-matrices $\sigma_{ab}^{(n)}$, namely $\sigma_{ab}^{(m)}\sigma_{bc}^{(n)} = \delta_{mn}\sigma^{(0)}_{ac} + \epsilon_{mno}\sigma^{(o)}_{ac}$, and their tracelessness $\sigma^{(m)}_{aa} = 0$, it is possible to invert the last relation and to obtain the expansion coefficients,
\begin{equation}
s_A = \frac{1}{2}\varphi_{A,ab}\sigma^{(0)}_{ab},
\quad
\epsilon_{+,A} = \frac{1}{2}\varphi_{A,ab}\sigma^{(1)}_{ab},
\mathrm{~and}\quad
\epsilon_{\times,A} = \frac{1}{2}\varphi_{A,ab}\sigma^{(3)}_{ab}.
\end{equation}

This path is motivated by the weak lensing shear $\gamma$, which results from the tensor $\psi_{B,ij}$ containing the second derivatives of the weak lensing potential $\psi_B$,
\begin{equation}
\psi_{B,ij} = \partial_i\partial_j\psi_B
\quad\mathrm{with}\quad
\psi_B = \int\dd\chi\:\Phi = \int\dd\chi\:W_{\psi,B}(\chi)\Phi.
\end{equation}
Again, there is an analogous decomposition
\begin{equation}
\psi_{B,ij} = \kappa_B\sigma^{(0)}_{ij} + \gamma_{+,B}\sigma^{(1)}_{ij} +\gamma_{\times,B}\sigma^{(3)}_{ij}
\end{equation}
with the analogous inversion,
\begin{equation}
\kappa_B = \frac{1}{2}\psi_{B,ij}\sigma^{(0)}_{ij},
\quad
\gamma_{+,B} = \frac{1}{2}\psi_{B,ij}\sigma^{(1)}_{ij},
\mathrm{~and}\quad
\gamma_{\times,B} = \frac{1}{2}\psi_{B,ij}\sigma^{(3)}_{ij}.
\end{equation}

This implies that the statistics of all modes of the shape and size field can be described by spectra of the source fields, which in turn are given by a Limber-projection. Specifically, the spectrum of $\varphi_{A,ab}$ reads
\begin{equation}
\bra\varphi_{A,ab}(\bmath\ell)\varphi_{B,ij}^*(\bmath\ell^\prime)\ket = 
(2\pi)^2\dirac(\bmath\ell-\bmath\ell^\prime)\:C^{\varphi_A\varphi_B}_{abij}(\ell)
\quad\mathrm{with}\quad
C^{\varphi_A\varphi_B}_{abij}(\ell) = 
\ell_a\ell_b\ell_i\ell_j\:\int\frac{\dd\chi}{\chi^2}\:W_{\varphi,A}(\chi)W_{\varphi,B}(\chi)\:P_{\Phi\Phi}(k = \ell/\chi),
\end{equation}
similarly, one obtains for the the field $\psi_{B,ij}$,
\begin{equation}
\bra\psi_{A,ab}(\bmath\ell)\psi_{B,ij}^*(\bmath\ell\prime)\ket = 
(2\pi)^2\dirac(\bmath\ell-\bmath\ell^\prime)\:C^{\psi_A\psi_B}_{abij}(\ell)
\quad\mathrm{with}\quad
C^{\psi_A\psi_B}_{abij}(\ell) = 
\ell_a\ell_b\ell_i\ell_j\:\int\frac{\dd\chi}{\chi^2}\:W_{\psi,A}(\chi)W_{\psi,B}(\chi)\:P_{\Phi\Phi}(k = \ell/\chi),
\end{equation}
and finally for their cross-correlation,
\begin{equation}
\bra\varphi_{A,ab}(\bmath\ell)\psi_{B,ij}^*(\bmath\ell^\prime)\ket =
(2\pi)^2\dirac(\bmath\ell-\bmath\ell^\prime)\:C^{\varphi_A\psi_B}_{abij}(\ell)
\quad\mathrm{with}\quad
C^{\varphi_A\psi_B}_{abij}(\ell) =
\ell_a\ell_b\ell_i\ell_j\:\int\frac{\dd\chi}{\chi^2}\:W_{\varphi,A}(\chi)W_{\psi,B}(\chi)\:P_{\Phi\Phi}(k = \ell/\chi).
\end{equation}


\spirou{add: Jeans-model}


% --- section:  --- %
\section{shape correlations of galaxies}\label{sect_shapes}
Now, the decomposition with Pauli-matrices makes it possible to write down all ellipticity spectra as contractions of the the possible spectra of the source terms, for lensing,
\begin{equation}
C^{\gamma\gamma}_{AB}(\ell) = \frac{1}{4}
\left(\sigma^{(1)}_{ab}\sigma^{(1)}_{ij} + \sigma^{(3)}_{ab}\sigma^{(3)}_{ij}\right)
C^{\psi_A\psi_B}_{abij}(\ell),
\end{equation}
for intrinsic alignments,
\begin{equation}
C^{\epsilon\epsilon}_{AB}(\ell) = \frac{1}{4}
\left(\sigma^{(1)}_{ab}\sigma^{(1)}_{ij} + \sigma^{(3)}_{ab}\sigma^{(3)}_{ij}\right)
C^{\varphi_A\varphi_B}_{abij}(\ell),
\end{equation}
and for the cross-correlation between the two,
\begin{equation}
C^{\epsilon\gamma}_{AB}(\ell) = \frac{1}{4}
\left(\sigma^{(1)}_{ab}\sigma^{(1)}_{ij} + \sigma^{(3)}_{ab}\sigma^{(3)}_{ij}\right)
C^{\varphi_A\psi_B}_{abij}(\ell).
\end{equation}
A measurement of the shape correlations is limited by a Poissonian shape noise contribution,
\begin{equation}
N_{AB}^\mathrm{shape}(\ell) = \sigma^2_\mathrm{shape}\frac{n_\mathrm{tomo}}{\bar{n}}\delta_{AB}.
\end{equation}


The extrinsic and intrinsic shape spectra are shown for a tomographic survey in Fig.~\ref{fig:shapeshape}.

\begin{figure}
\centering
%\includegraphics[scale=0.1]{wip.jpg}
\caption{Shape-shape correlations as a function of $\ell$. \BG{To plot: $C_{ij}^{\gamma\gamma}(\ell)$,$C_{ij}^{\gamma\epsilon}(\ell)$,$C_{ij}^{\epsilon\epsilon}(\ell)$ and $\sigma_{\epsilon^2}/n$}}
\label{fig:shapeshape}
\end{figure}


% --- section:  --- %
\section{size correlations of galaxies}\label{sect_sizes}
In a similar manner like in the previous section, one obtains the size spectra from contracting the possible spectra of the source terms, for lensing,
\begin{equation}
C^{\kappa\kappa}_{AB}(\ell) = \frac{1}{4}\sigma^{(0)}_{ab}\sigma^{(0)}_{ij}C^{\psi_A\psi_B}_{abij}(\ell),
\end{equation}
for intrinsic alignments,
\begin{equation}
C^{s\kappa}_{AB}(\ell) = \frac{1}{4}\sigma^{(0)}_{ab}\sigma^{(0)}_{ij}C^{\varphi_A\psi_B}_{abij}(\ell),
\end{equation}
and again, for the cross-correlation between the two,
\begin{equation}
C^{ss}_{AB}(\ell) = \frac{1}{4}\sigma^{(0)}_{ab}\sigma^{(0)}_{ij}C^{\varphi_A\varphi_B}_{abij}(\ell).
\end{equation}
In the estimation process, there is a constant, diagonal noise contribution
\begin{equation}
N_{AB}^\mathrm{size}(\ell) = \sigma^2_\mathrm{size} \frac{n_\mathrm{tomo}}{\bar{n}}\delta_{AB}
\end{equation}


Fig.~\ref{fig:shapeshape} shows the intrinsic and extrinsic size-spectra, as they would result from a tomographic survey.

\begin{figure}
\centering
%\includegraphics[scale=0.1]{wip.jpg}
\caption{Size-size correlations as a function of $\ell$. \BG{To plot: $C_{ij}^{\kappa\kappa}(\ell)$,$C_{ij}^{\kappa s}(\ell)$,$C_{ij}^{ss}(\ell)$ and $\sigma_{s^2}/n$}}
\label{fig:shapeshape}
\end{figure}


\spirou{add: consistency relations}

\spirou{add: cross-terms between shape and size are zero}

\spirou{add: show proof, statistically isotropic fields}



% --- section:  --- %
\section{information content of shape and size correlations}\label{sect_fisher}

\spirou{construction of the covariance}

$$\varmathbb{C}=\begin{pmatrix}
C_{ij}^{\gamma\gamma}+C_{ij}^{\epsilon\epsilon}+2C_{ij}^{\gamma\epsilon}+\frac{\sigma_{\epsilon^2}}{n} & C_{ij}^{\gamma s}+C_{ij}^{\epsilon\kappa}+C_{ij}^{\gamma\kappa}+C_{ij}^{\epsilon s}   \\
C_{ij}^{\gamma s}+C_{ij}^{\epsilon\kappa}+C_{ij}^{\gamma\kappa}+C_{ij}^{\epsilon s} & C_{ij}^{\kappa\kappa}+C_{ij}^{ss}+2C_{ij}^{\kappa s}+\frac{\sigma_{s^2}}{n}
\end{pmatrix}$$

The Fisher-matrix $F_{\mu\nu}$ for a tomographic survey assumes the form
\begin{equation}
F_{\mu\nu} = \sum_\ell\frac{2\ell+1}{2}\mathrm{tr}\left(\partial_\mu\ln C\:\partial_\nu\ln C\right)
\end{equation}
where we implicitly assume a full sky coverage by having independent Fourier-modes, but scale down the Fisher-matrix with the fraction $f_\mathrm{sky}=0.5$ for the Euclid survey.

As the shape and size spectra are equal and carry the identical dependence on cosmological parameters, as they are essentially contractions of the same source terms for lensing, intrinsic alignments and their cross-correlations, the covariance matrix of the combined measurement is identical to that of an individual measurement with a rescaled noise term,
\begin{equation}
\frac{1}{\sigma^2_\mathrm{tot}} = \frac{1}{\sigma^2_\mathrm{shape}} + \frac{1}{\sigma^2_\mathrm{size}}.
\end{equation}


Fig.~\ref{fig:fisher} shows a constraints on a $w$CDM-cosmology from galaxy shapes and galaxy sizes.

\begin{figure}
\centering
%\includegraphics[scale=0.1]{wip.jpg}
\caption{Fisher ellipses: size only, shape only, size+shape combined, D fixed, LCDM parameters set}
\label{fig:fisher}
\end{figure}


Fig.~\ref{fig:fisher2} illustrates, how much information is added to the lensing data by including intrinsic alignments, again separated by galaxy shape spectra, size spectra and both combined.

\begin{figure}
\centering
%\includegraphics[scale=0.1]{wip.jpg}
\caption{Fisher ellipses: size only, shape only, size+shape combined, D variable, LCDM parameters set}
\label{fig:fisher2}
\end{figure}


\begin{table}
\begin{center}
\begin{tabular}{cc}
\hline\hline\\
\hline
\end{tabular}
\caption{Statistical precision $\sigma_D$ for measuring the alignment parameter $D$ of the linear alignment model.}
\label{tab_precision}
\end{center}
\end{table}


% --- section: summary --- %
\section{summary}\label{sect_summary}



% --- section: acknowledgements --- %
\section*{Acknowledgements}
BG thanks...   BMS likes to thank the Universidad del Valle in Cali, Colombia, for their kind hospitality.


% --- bibliography --- %
\bibliographystyle{mnras}
\bibliography{references}


\bsp
\label{lastpage}
\end{document}
